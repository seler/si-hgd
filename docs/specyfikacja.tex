% generated by Docutils <http://docutils.sourceforge.net/>
\documentclass[a4paper,english]{article}
\usepackage{fixltx2e} % LaTeX patches, \textsubscript
\usepackage{cmap} % fix search and cut-and-paste in PDF
\usepackage[T1]{fontenc}
\usepackage[utf8]{inputenc}
\usepackage{ifthen}
\usepackage{babel}
\usepackage{tabularx}

%%% Custom LaTeX preamble
% PDF Standard Fonts
\usepackage{mathptmx} % Times
\usepackage[scaled=.90]{helvet}
\usepackage{courier}

%%% User specified packages and stylesheets

%%% Fallback definitions for Docutils-specific commands

% providelength (provide a length variable and set default, if it is new)
\providecommand*{\DUprovidelength}[2]{
  \ifthenelse{\isundefined{#1}}{\newlength{#1}\setlength{#1}{#2}}{}
}

% docinfo (width of docinfo table)
\DUprovidelength{\DUdocinfowidth}{0.9\textwidth}

% hyperlinks:
\ifthenelse{\isundefined{\hypersetup}}{
  \usepackage[unicode,colorlinks=true,linkcolor=blue,urlcolor=blue]{hyperref}
  \urlstyle{same} % normal text font (alternatives: tt, rm, sf)
}{}
\hypersetup{
  pdftitle={Sztuczna inteligencja - Specyfikacja projektu},
}

%%% Body
\begin{document}

% Document title
\title{Sztuczna inteligencja - Specyfikacja projektu%
  \phantomsection%
  \label{sztuczna-inteligencja-specyfikacja-projektu}%
  \\ % subtitle%
  \large{Temat: ``Hierarchiczne grupowanie danych''}%
  \label{temat-hierarchiczne-grupowanie-danych}}
\author{}
\date{}
\maketitle

% Docinfo
\begin{center}
\begin{tabularx}{\DUdocinfowidth}{lX}
\textbf{Autorzy}: &
Rafał Selewońko, Paweł Tomkiel
\\
\textbf{Grupa}: &
PS8
\\
\textbf{Prowadzący}: &
mgr inż. Dariusz Małyszko
\\
\end{tabularx}
\end{center}

% spis treści::


%___________________________________________________________________________

\section*{Wprowadzanie i pobieranie danych%
  \phantomsection%
  \addcontentsline{toc}{section}{Wprowadzanie i pobieranie danych}%
  \label{wprowadzanie-i-pobieranie-danych}%
}

Projekt uruchamia się widokiem pod-aplikacji pozwalającej na wybór
katalogu, w którym przechowywane są pliki danych. Wszystkie pliki są
wylistowane w lewej części okna i można wybrać jeden z nich do
wyświetlenia w jego prawej części. Istnieje osobny przycisk do
załadowania katalogu i osobny do wyświetlenie pliku.


%___________________________________________________________________________

\section*{Pobieranie parametrów zadania%
  \phantomsection%
  \addcontentsline{toc}{section}{Pobieranie parametrów zadania}%
  \label{pobieranie-parametrow-zadania}%
}

Po załadowaniu pliku będzie możliwe(projekt jest nadal w fazie
budowania) uruchomienie aplikacji głównej zajmującej się obliczeniami.
Aktualnie wczytany plik w pod-aplikacji jest przekazywany do aplikacji
głównej. Następnie będzie przetwarzany zgodnie z algorytmem grupowania
hierarchicznego danych i będą wyświetlane oraz wizualizowane wyniki
obliczeń.


%___________________________________________________________________________

\section*{Obliczenia%
  \phantomsection%
  \addcontentsline{toc}{section}{Obliczenia}%
  \label{obliczenia}%
}

Algorytm grupujący hierarchicznie dane opiera się na kilku prostych
krokach:
%
\begin{quote}{\ttfamily \raggedright \noindent
1.~Tworzy~m~grup~zawierających~po~jednym~obiekcie\\
2.~Wyznacza~odległości~między~grupami\\
3.~Znajduje~parę~najbliższych~grup~i~łączy~je~w~jedną~grupę\\
4.~Wyznacz~jeszcze~raz~odległości~między~nową~grupą~a~pzoostałymi~grupami\\
5.~Powtarzaj~kroki~aż~do~momentu~gdy~pozostanie~zadana~liczba~grup
}
\end{quote}


%___________________________________________________________________________

\section*{Finalizacja%
  \phantomsection%
  \addcontentsline{toc}{section}{Finalizacja}%
  \label{finalizacja}%
}

Po przeprowadzeniu wszystkich obliczeń i pogrupowaniu obiektów projekt
będzie wizualizował(w osobnym oknie) w przestrzeni dwuwymiarowej
odległości między
obiektami i będzie je łączył w grupy. Forma tekstowa wyników obliczeń
też jest w planach, najpewniej w formie tabeli.

\end{document}
